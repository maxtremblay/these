\documentclass[titlepage,openright,letterpaper,12pt]{book}
\usepackage{config/entete}
\usepackage{config/commandes} % Load les packages et définit des commandes.

%=============================================================================%

% On crée des bools qui sont False par défaut.
\newtoggle{VersionLivre}
\newtoggle{LivrePGChVide}
\newtoggle{ForceEntete}
\newtoggle{AuteureFemme}
\newtoggle{MemoirePasThese}
\newtoggle{useCustomFonts}
\newtoggle{generatePDFa}
\newtoggle{IntroConcluSansNombre}

% Switchboard, l'endroit ou on ajuste les toggles.
% Décommenté -> True; commenté -> False

%\toggletrue{VersionLivre}          % Décommenter pour faire la version livre
%\toggletrue{LivrePGChVide}         % Page gauche de chapiter vide en mode livre.
%\toggletrue{ForceEntete}            % Entête même pour version électronique.
%\toggletrue{AuteureFemme}          % Décommenter si l'auteur est une femme.
%\toggletrue{MemoirePasThese}       % Décommenter dans le cas d'un mémoire.
\toggletrue{IntroConcluSansNombre}  % Intro et conclusion non numérotées
\toggletrue{useCustomFonts}         % Fonts différents, voir switchboard.tex.
%\toggletrue{generatePDFa}           % Génère un PDFa plutôt qu'un PDF standard.

%=============================================================================%

\title{
    Une odyssée de la communication classique à la tolérance aux fautes quantique
} % Pour la page de titre/jury.
\author{Maxime Tremblay}  % Idem
\Organisation{UNIVERSITÉ de SHERBROOKE}
\Location{Sherbrooke, Québec, Canada}
\ResumeCourt{}
\date{\today}       % Idem
\MotsClefs{Mots-Clefs \sep Pertinents \sep avec séparateurs}

% On cache le code exécuté dans ce fichier parce qu'il est laid et impertinent.
\input{config/switchboard.tex}

%=============================================================================%

\begin{document}

%=============================================================================%
% Titre
\input{sources/titlepage.tex}

\frontmatter % Pagination de préambule

% Jury
\begin{comment}
\end{comment}
\makeatletter   % Permet d'accèder aux variables @

\iftoggle{LivrePGChVide}
{}
{
    % Next two lines force a linebreak in ebook versions
    \chapter*{}
    \vspace{-4.7cm}
}

\thispagestyle{empty}

\begin{center}
    \vglue 2cm
    % Le \underline{\hspace{5cm}}\\ 
    %Le \@date % Lorsque le document sera accepté!
    \vspace{2cm}
    \scalebox{1} % Empêche le retour à la ligne si le nom est trop long.
    % {\it le jury a accepté \leDocument\ de \monsieurMadame~\@author~dans sa version finale.} 
    
    \vspace{1cm}
    Membres du jury\\
    \vspace{1cm}

    Professeur Stefanos Kourtis\\
    Directeur de recherche\\
    Département de physique\\
    \vspace{1cm}

    Professeur Guillaume Duclos-Cianci\\
    Co-directeur de recherche\\
    Département de physique\\
    \vspace{1cm}
    
    Professeur Alexandre Blais\\
    Membre interne\\
    Département de physique\\
    \vspace{1cm}

    Professeur David Sénéchal\\
    Président rapporteur\\
    Département de physique\\
    \vspace{1cm}

    Professeur Gilles Zémor\\
    Membre externe\\
    Université de Bordeaux\\
    \vspace{1cm}
\end{center}

%\clearpage

\makeatother    % Plus d'accès aux variables @


% Dédicace
\input{sources/dedicace.tex}
\thispagestyle{empty}

% Sommaire
\clearpage  % Mets le sommaire à la bonne page dans la TOC
\chapter*{Sommaire}
\addcontentsline{toc}{chapter}{Sommaire}
\begin{comment}
\end{comment}

C'est ouvrage traite principalement de la protection de
l'information. 
Pas au sens que nous entendons souvent dans les médias de protection des renseignements privés,
mais plutôt au sens de robustesse face à la corruption des données.
En effet,
lorsque nous utilisons un cellulaire pour envoyé un texto,
plusieurs facteurs, 
comme les particules atmosphériques et l'interférence avec d'autres signaux,
peuvent modifier le message initial.
Si nous ne faisons rien pour protéger le signal,
il est peu problable que le contenu du texto reste inchangé
lors de la réception.

C'est ce problème qui a motivé le premier projet 
de recherche de cette thèse.
Sous la supervision du professeur David Poulin,
j'ai étudié une généralisation des codes polaires,
une technologie au coeur du protocol de communication de 5ième génération (5G).
Pour cela,
j'ai utilisé les réseaux de tenseurs, 
des outils mathématiques initialement développé pour étudier
les matériaux quantiques.
L'avantage de cette approche est qu'elle permet une représentation
graphique intuitive du problème, 
ce qui facilite grandement le développement des algorithmes.

Cette idée d'utiliser des outils mathématiques graphiques pour 
étudier des problèmes de protection de l'information
sera le fil conducteur pour le reste de la thèse.
Cependant, 
pour la suite, 
les erreurs n'affecteront plus des systèmes de communications classiques,
mais plutôt des systèmes de calcul quantique.
Et, comme nous le verrons dans cette thèse,
les systèmes quantiques sont naturellement beaucoup plus sensible aux erreurs.

À cet effet, 
j'ai effectué un stage au sein de l'équipe de Microsoft Research,
principalement sous la supervision de Micheal Beverland, 
lors duquel j'ai conçu des circuits permettant de mesurer un système quantique afin d'identifier
les potentiels fautes qui affectent celui-ci.
J'ai également proposée une architecture qui permetterait d'implémenter ces circuits
de façon réaliste en laboratoire.
Finalement, avec le reste de l'équipe, 
nous avons prouver mathématiquement que les circuits que j'ai développés sont optimaux.
L'ensemble de ces résultats sont inspirés de méthodes issues de la théorie des graphes.

J'ai terminé ma thèse sous la supervision du professeur Stefanos Kourtis.
Avec celui-ci,
j'ai créé une méthode, toujours basée sur une approche graphique, 
qui permet d'automatiquement concevoir de nouveaux protocoles de 
corrections des erreurs dans un système quantique.
Cela nous a permis de montrer qu'il est probablement beaucoup plus facile 
que ce que le croyait la communauté scientifique de concevoir de tels protocoles.

Vous aurez remarqué que lors de tous ces projets,
je n'ai jamais eu le même superviseur.
C'est pour cela que je qualifie ma thèse d'odyssée.
Celle-ci a été parsemée d'embuches.
D'abord avec le triste départ trop rapide du professeur David Poulin.
C'est suite à cela que j'ai postulé pour un stage au sein de l'équipe de Microsoft
dans le but d'avoir une nouvelle supervision.
Cependant, le stage c'est finalement déroulé en virtuel après avoir été repousssé
plusieurs fois en raison de la pandémie.
C'est après ce stage que je me suis finalement greffé à l'équipe du professeur Kourtis 
qui venait de démarrer son groupe.
Heureusement,
lors de toutes cette étape, 
je pouvais toujours compter sur le soutien de Guillaume Duclos-Ciani,
initialement professionel de recherche au sein du groupe de David Poulin.

En bref,
j'espère que vous aurez du plaisir à lire cette thèse même si celle-ci 
tire un peu dans tous les sens.
Elle réflète les connaissances et compétences que j'ai acquéries en travaillant sous 
la supervision de plusieurs mentors en plus de s'intéresser à l'un des problèmes
parmi les plus importants pour la réalisation des promesses de l'informatique quantique.

Bonne lecture!


% Remerciements
\chapter*{Remerciements}
\begin{comment}
\end{comment}

\chapter*{Remerciements}

Cette thèse conclut mon parcours à l'université de Sherbrooke.
Ce parcours à débuter il y a cinq ans,
lorsque j'ai cogné à la porte de David Poulin,
comme ça, sans rendez-vous.
Après une discussion qui s'est étirée plus que je l'imaginais,
je suis sorti de son bureau avec un nouveau projet, 
réaliser une maitrise en informatique quantique.
J'ai commencé peu de temps après et j'ai rapidement réalisé que David 
était un chercheur d'exception en plus d'être un excellent pédagogue et mentor.
Merci pour la chance que tu m'as offerte et ton encadrement.
Travailler avec toi à grandement influencer le scientifique que je suis devenu.

Dans mon doctorat,
j'ai également eu la chance d'effectuer un stage au sein de l'équipe de Microsoft.
J'aimerais remercier toute l'équipe et plus particulièrement mon mentor,
Micheal Berverland, qui a réussi à me redonner la motivation envers la recherche
que je commençais à perdre après le début de la pandémie et le départ de David qui est
arrivé presque simultanément.
Merci d'avoir éveillé à nouveau ma curiosité scientifique.

Finalement,
j'ai terminé mon doctorat sous la supervision de Stefanos Kourtis.
Merci de m'avoir accueilli dans ton groupe à la suite de mon stage alors que j'errais sans direction.
J'aimerais particulièrement souligner ton optimiste et ton ambition scientifique.
Après avoir travaillé sous ta supervision,
je quitte avec la conviction que si l'on croit assez fort à un résultat,
celui-ci finira toujours pas se concrétiser.
Je te remercie également pour la relation amicale et décontractée que nous entretenons.

Je remercie aussi Guillaume Duclos-Cianci qui m'a accompagné pour la majeure partie de mes études.
J'apprécie grandement ton soutien autant sur les plans administratifs, 
scientifiques et personnels.

Je dis également merci à Jacques Boulanger qui a pris le temps de relire ma thèse afin de réduire
significativement le nombre de fautes de grammaires et d'orthographe. C'est un coup de main très apprécié.  

J'aimerais également remercier toutes les personnes que j'ai rencontrées lors de mon parcours.
Un merci tout spécial à Benjamin, Jessica, Oumar, Claude avec qui
je me suis lié d'amitié.
Un merci également aux gens avec qui j'ai eu la chance d'avoir des échanges scientifiques
de grandes qualités, soient Anirudh, Nouédyn, Martin et Christopher.
Il y a bien évidemment des personnes qui auraient pu se retrouver dans les deux groupes
et d'autres personnes que je n'ai pas nommées qui méritent également mes meilleurs remerciements. 

Finalement,
je remercie ma famille, Alexis, François, Frédérick et Sylvie,
ainsi que mes amis et amies des longues dates, Alexandre, Cedrick, Cédrick, Charles-Alexandre,
Daehli, Félix, Frédéric, Gabriel, Jérémie, Jonathan, Kevin, Maxime-Ève et Philippe.
Tous ces gens m'accompagnent et me supportent depuis toujours.

Je garde un dernier merci tout spécial pour Marie-Eve Boulanger.
Je suis infiniment chanceux de toujours pouvoir compter sur toi,
tu es une personne exceptionnelle.
Maintenant que les études sont terminées,
j'ai très hâte de commencer ce nouveau projet de parentalité avec toi.



% Tables des matières/Figures
{
    \setlength{\parskip}{0ex}
    \tableofcontents
    \listoffigures
    %\listoftables
}

%=============================================================================%

\mainmatter % Pagination standard
\onehalfspacing

%-----------------------------------------------------------------------------%

% \include serait préférable à \input à partir d'ici, mais MiKTeX sur Windows
% n'aime pas les \cite dans un \include. Fonctionne parfaitement sur TeXLive.

\begin{comment}
\end{comment}

\Introduction   % Chapitre qui ne sera pas numéroté si IntroConcluSansNombre est Vrai

L'ordinateur et l'internet figurent parmi les technologies qui ont le plus impacté
notre mode de vie et l'organisation de nos sociétés.
En effet,
il est désormais possible d'accéder à une quantité phénoménale de connaissance,
de connecter avec une personne de l'autre côté du globe
et d'automatiser les taches du quotidient.
Tout cela avec une efficience qui peut sembler sans limite.

Cependant, 
bien que ces technologies offrent des performances impressionantes,
il existe des taches pour lesquelles le temps nécessaire peut s'approche ou
dépasse l'age actuel de l'univers.
Bien sur,
le temps d'exécution dépend de la taille de la tache a effectuer.
Par contre, 
la difficulté de différentes taches n'augmente pas de manière similaire.

Par exemple,
il sera deux fois plus long pour un ordinateur d'inverser l'ordre d'une liste 
si la longueur de cette dernière est doublée.
Par contre,
il ne suffit que d'ajouter un chiffre à un nombre pour que celui-ci soit deux 
fois plus long à factoriser~\cite{arora_computational_2009}.
Formellement,
on dit que la complexité du premier problème augmente linéairement avec la taille de l'entrée,
alors que celle du second problème augmente exponentiellement.

Les problèmes dont la complexité augmente exponentiellement requierent donc un temps de calcul
hors de porté, et ce, même pour des instances de tailles modestes.
Par contre,
cette façon de calculer la complexité suppose un modèle de calcul dit classique,
soit le modèle utilisé par les ordinateurs actuels.
Et,
comme vous vous en doutez probablement si vous lisez cette thèse,
le modèle de calcul classique n'est pas le seul modèle de calcul.

L'idée du calcul quantique est attribuée à Ed Fredkin et Richard Feynman et 
c'est ce dernier qui aurait présenté cette possibilité en 1981 lors d'une conférence au MIT~\cite{hoofnagle_birth_2021}.
Depuis,
le calcul quantique a fait beaucoup de chemin,
nottament après une publication de Peter Shor dans laquelle il introduit un 
algorithme pouvant factoriser un nombre avec une complixité polynomiale selon
le nombre de chiffres à l'aide d'un ordinateur quantique~\cite{shor_algorithms_1994}.
Dans ce cas,
doubler la longueur du nombre de fait que multiplier par huit le temps de calcul.
Cela est beaucoup plus efficace que la croissance exponentielle requise par un ordinateur classique.

Bien que ce résultat ne permet de résoudre tous les problèmes dont la complexité classique
est exponentielle,
il démontre tout de même que certains problèmes qui semblaient hors d'atteintes des ordinateurs
classiques seraient en théorie résolubles, en un temps raisonnable, par un ordinateur quantique.





\include{sources/theorie.tex}
\include{sources/resultats.tex}
\begin{comment}
\end{comment}

\Conclusion % Chapitre qui ne sera pas numéroté si IntroConcluSansNombre est Vrai

Dans cette thèse,
j'ai présenté trois projets de recherche effectués lors de mon doctorat.
Tous ces projets traitent de la correction des erreurs,
mais dans des modèles de calcul différents.

Lors du premier projet,
j'ai mesuré l'impact de divers paramètres sur la performance et le temps de décodage
des codes polaires convolutifs.
Cela a permis d'identifier que les codes de profondeur et largeur deux offrent les
meilleures performances en fonction du temps de décodage.
Ce résultat pousse donc plus loin les performances des codes polaires standards
qui sont présentement utilisés dans le réseau de communication 5G.
Bien que les considérations d'ingénierie d'un réseau de communication sortent du cadre de cette thèse,
il est évident que toutes les améliorations à la correction d'erreurs peuvent réduire les 
couts d'implémentation d'un tel réseau.
En effet,
comme le taux d'erreur dépend de la distance entre les antennes,
un code qui permet de corriger plus d'erreurs permet également d'augmenter la distance
entre les antennes ce qui permet d'en réduire le nombre.

Le second projet que j'ai présenté est en réalité le dernier que j'ai effectué.
Dans le cadre de celui-ci,
je me suis intéressé à la construction de codes correcteurs quantiques.
La contribution la plus importante de ce projet réside dans la connexion
entre les problèmes de satisfaction de contraintes et la construction de codes.
Cela a permis de démontrer numériquement l'existence d'un seuil de satisfaisabilité
pour la probabilité d'inclusion d'une arête dans le graphe de support,
au-delà duquel il est facile de trouver des codes.
Il existe donc des régimes de paramètres pour lesquels il est aisé de construire des codes.
De plus,
j'ai poussé la méthode plus loin pour construire une famille de codes LDPC offrant des 
performances optimales pour le canal à effacement.
Il reste encore du chemin à parcourir,
mais cette approche offre la possibilité de découvrir des codes correcteurs pouvant être
adaptés aux contraintes physiques des ordinateurs quantiques à court et moyen terme.

Le troisième projet dont j'ai traité s'intéresse à la réalisation d'une architecture
permettant l'implémentation d'une mémoire quantique.
Des simulations numériques montrent également que cette approche permet de réduire significativement
le nombre de qubits nécessaires.
Bien sûr,
il reste de nombreux défis à résoudre pour implémenter cette architecture en laboratoire,
notamment l'implémentation de connexions à longue portée entre les qubits.
Néanmoins,
l'approche présentée permet d'éliminer les croisements entre les connexions en exploitant
un petit nombre de couches dans la troisième dimension.
Enfin,
les méthodes de traçage de graphes utilisées pour obtenir cette architecture 
ont le potentiel d'être appliquées pour contourner d'autres limitations à l'implémentation
du calcul tolérant aux fautes.

Dans l'ensemble de ces projets,
les modèles de bruits étudiés demeurent agnostiques de la technologie utilisée pour construire
les qubits.
Ce choix est fait puisqu'il est encore trop tôt pour choisir une technologie parmi toutes celles
qui sont proposées.
Il reste donc du travail à faire pour adapter les méthodes proposées dans cette thèse aux diverses 
technologies de qubits.
Cependant,
les travaux de cette thèse démontrent bien comment il est possible d'utiliser divers outils,
des réseaux de tenseurs aux méthodes de traçage de graphes en passant par les problèmes de satisfaction
de contraintes,
pour faciliter le design de systèmes quantiques tolérants aux fautes.

%-----------------------------------------------------------------------------%

\singlespacing

\begin{comment}
\end{comment}

\appendix
\renewcommand\chapterstring{Annexe}

\chapter{Complexité de calcul}
\label{chap:complexite_calcul}

\begin{itemize}
  \item Notation $O(\cdot)$ et $\Omega(\cdot)$
\end{itemize}

\chapter{Théorie des graphes}
\label{chap:theo_graphe}

\chapter{Théorie de l'information}
\label{chap:theo_info}

\begin{itemize}
  \item Entropie
\end{itemize}





%=============================================================================%

% Voir switchboard.tex pour le bibliographystyle selon le type de document.
\bibliography{references}        % Le fichier de bibliographie est memoire.bib.

\end{document}
