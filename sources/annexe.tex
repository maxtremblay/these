\begin{comment}
\end{comment}

\appendix
% \renewcommand\chapterstring{Annexe}

\chapter{Théorie de l'information}
\label{chap:theo_info}

Considérant une variable aléatoire $X$ dont les valeurs possibles sont les éléments de $\mathcal X$
avec probabilité $\Pr: \mathcal X \to [0, 1]$,
la quantité d'information acquise après une réalisation $x \in \mathcal X$ de cette variable
est $\log(1/\Pr(x))$.
L'\textbf{entropie de Shannon} est l'espérance de cette valeur, soit 
\begin{align}
  H(X) = \mathbb E\qty(\log\qty(\frac{1}{\Pr(X)})) = -\sum_{x\in\mathcal X}\Pr(x) \log(\Pr (x)).
\end{align}
Dans cette définition,
il est considéré que $p\log(p) = 0$ lorsque $p = 0$.

Pour deux variables aléatoires $X, Y$ de domaines $\mathcal X, \mathcal Y$,
l'\textbf{entropie conjointe} est 
\begin{align}
  H(X, Y) 
  = \mathbb E\qty(\log\qty(\frac{1}{\Pr(X, Y)})) 
  = -\sum_{x\in\mathcal X, y \in\mathcal Y}\Pr(x, x) \log(\Pr(x, y))
\end{align}
et l'\textbf{entropie conditionnelle} est
\begin{align}
  H(X | Y) 
  = \mathbb E\qty(\log\qty(\frac{1}{\Pr(X|Y)})) 
  = -\sum_{x\in\mathcal X, y \in\mathcal Y}\Pr(x, x) \log(\Pr(x|y)).
\end{align}
Il est aisé de vérifier que 
\begin{align}
  H(X, Y) = H(X | Y) + H(Y) = H(Y | X) + H(X).
\end{align}
Ainsi,
l'information acquise après la réalisation simultanée de deux variables aléatoires
est équivalente à la somme de l'information acquise en réalisant la première 
et de l'information acquise en réalisant la seconde en connaissant la valeur de la première.

L'\textbf{information mutuelle} entre deux variables aléatoires $X, Y$ est
l'information que révèle une réalisation d'une des variables sur l'autre variable.
L'information mutuelle se mesure en comparant la probabilité conjointe de $X, Y$
aux probabilités marginales, soit
\begin{align}
  I(X ; Y) 
  &= \mathbb E\qty(
    \log\qty(\frac{1}{\Pr(X)\Pr(Y)})
    -
    \log\qty(\frac{1}{\Pr(X, Y)})
  ) \notag \\
  &= -\sum_{x\in\mathcal X, y \in\mathcal Y}\Pr(x, x) \log(\frac{\Pr(X, Y)}{\Pr(X)\Pr(Y)}).
\end{align}
Cette définition est symétrique,
c'est-à-dire $I(X ; Y) = I(Y ; X)$.
De plus,
l'information mutuelle est reliée aux diverses entropies selon
\begin{align}
  I(X;Y)
  = H(X) - H(X | Y)
  = H(Y) - H(Y | X)
  = H(X) + H(Y) - H(X, Y).
\end{align}
Le première équation illustre bien que l'information mutuelle correspond à la différence
d'information acquise par une réalisation de $X$ lorsque $Y$ est inconnue ou connue.


\chapter{Théorie des graphes}
\label{chap:theo_graphe}

Un \textbf{graphe} est une paire $(S, A)$ telle que 
\begin{align}
  A \subseteq \qty{\qty{s, t} : s,t \in S}.
\end{align}
Les éléments de $S$ se nomment \textbf{sommets}
et les éléments de $A$ se nomment \textbf{arêtes}.
Deux sommets $s, t \in S$ sont \textbf{connectés} si $\qty{s, t} \in A$.
Le \textbf{voisinage} d'un sommet $s$ est l'ensemble des arêtes contenant $s$,
soit 
\begin{align}
  \eta(s) = \qty{a \in A : s \in a}.
\end{align}
Le \textbf{degré} d'un sommet est la taille de son voisinage.

Un \textbf{graphe biparti} est un graphe pour lequel il existe une 
partition des sommets $S = U \cup V$ telle que 
\begin{align}
  A \subseteq \qty{\qty{u, v} : u \in U,\, v \in V}.
\end{align}
Ainsi, les arêtes sont restreintes entre les paires de sommets n'appartenant pas
au même ensemble.
Un \textbf{hypergraphe} est une généralisation d'un graphe
où le nombre de sommets par arête est arbitraire.
Ainsi,
un hypergraphe est une paire $(S, A)$ telle que
\begin{align}
  A \subseteq \qty{X \subseteq S}.
\end{align}
Le voisinage et le degré d'un sommet sont définis de façon similaire à un graphe.
De plus, le poids $|a|$ de $a \in A$ est le nombre de sommets de cette arête.

Il existe une correspondance entre les graphes bipartis et les hypergraphes.
Pour un hypergraphe $H = (S, A)$,
la fonction 
\begin{align}
  \phi(H) = (S \cup A, B),
\end{align}
avec $\qty{s, a} \in B$ si $s \in a$,
est bijective et associe à chaque hypergraphe un unique graphe biparti.

Un \textbf{cycle} de longueur $n$ est un graphe $C_n = (S, A)$
avec $S = \qty{s_1, s_2, \ldots, s_n}$ et 
$A = \qty{\qty{s_1, s_2}, \qty{s_2, s_3}, \ldots, \qty{s_n, s_1}}$.
Un cycle de longueur paire est un graphe biparti.

Soit deux graphes $G = (S, A)$ et $H = (T, B)$,
le \textbf{produit cartésien} $G \times H$ est un graphe $(S \times T, C)$
tel que $\qty{(s_1, t_1), (s_2, t_2)} \in C$ si 
$s_1 = s_2$ et $\qty{t_1, t_2} \in B$ ou si $t_1 = t_2$ et $\qty{s_1, s_2} \in A$.
Le produit cartésien de deux graphes contenant chacun deux sommets et une arête reliant ces
sommets est un cycle de longueur quatre.
Ainsi,
pour le produit cartésien $G \times H$,
chaque paire d'arêtes $a \in A$ et $b \in B$ engendre un cycle de longueur quatre dans $C$.

\chapter{Complexité de calcul}
\label{chap:complexite_calcul}

Dans cette section,
la rigueur des définitions est variable.
Définir formellement la complexité de calcul demande d'introduire un modèle de 
calcul comme la machine de Turing.
Cependant,
cela sort de beaucoup du cadre de cette thèse
et je me contente de définitions qui sont suffisantes à la compréhension du contenu de la thèse
et j'évite plusieurs détails.

Un \textbf{problème de calcul} est représenté par une relation $R \subseteq X \times Y$, 
où $X$ est l'ensemble des \textbf{entrées} et $Y$ est l'ensemble des \textbf{solutions}.
Les solutions de $x \in X$ sont les éléments de l'ensemble $Y_R(x) = \qty{y \in Y : (x, y)\in R}$.
Un \textbf{algorithme} $a: X \to Y$ résout $R$ si $a(x) \in Y_R(x)$ pour tous $x \in X$.

Un algorithme se décompose en une série d'opérations $a_1, \ldots a_c \in \mathcal A$ telle
que $a(x) = (a_c \circ \ldots \circ a_1)(x)$ où $\circ$ représente la composition
de fonctions.
Cette décomposition de $a$ varie en fonction de la taille de $x$
et dépend également des opérations accessibles $\mathcal A$.
Plus précisément,
il est supposé que les entrées $X$ sont décrites par des listes de tailles variées
d'éléments d'un alphabet $\mathcal X$.
Ainsi,
$X \subseteq \mathcal X^* = \cup_{n=0}^\infty \mathcal X^n$ et
une entrée $x \in X$ a une taille $n$ si $x \in \mathcal X^n$.

La \textbf{complexité} d'un algorithme est le nombre $c(n)$ d'opérations nécessaires
pour calculer $a(x)$ selon la taille $n$ de $x$.
Il est souvent difficile et peut pertinent de calculer la valeur exacte de $c(n)$
et il est généralement suffisant de connaitre le comportement général de $c(n)$.
Pour ce faire,
nous utilisons la \textbf{notation grand $\mathcal O$}.
Une fonction $f(n)$ à la complexité $\mathcal O(g(n))$ s'il existe
une constante $k$ telle que $f(n) \leq k g(n)$.
Généralement,
les complexités considérées sont les complexités \textbf{logarithmiques} $\mathcal O(\log n)$,
\textbf{linéaire} $\mathcal O(n)$,
\textbf{polynomiale} $\mathcal O(n^t)$ pour un entier $t > 0$
et \textbf{exponentielle} $\mathcal O(2^n)$.

Par exemple,
le problème du tri d'une liste d'entiers est défini par la relation
$R \subseteq \mathbb Z^* \times \mathbb Z^*$
où chaque paire $(x, y) \in R$ est une liste $x$ et sa version triée $y$.
Les opérations disponibles pour trier une liste sont la permutation et
la comparaison de deux éléments.
Comme il y a $n(n-1)/2$ paires de nombre,
un algorithme naif de tri a une complexité $\mathcal O(n^2)$.

Dans cette thèse,
je m'utilise trois classes de complexité de problème importantes.
Premièrement,
la \textbf{classe P} est l'ensemble des problèmes
pour lesquels il existe un algorithme de complexité polynomiale.
Deuxièmement,
la \textbf{classe NP} est l'ensemble des problèmes
pour lesquels il existe un algorithme de complexité polynomiale
qui permet de vérifier qu'une paire $(x, y)$ satisfait la relation.
La différence avec la classe P est qu'il n'est pas nécessaire qu'un 
algorithme de complexité polynomial soit en mesure de trouver $y \in Y_R(x)$,
mais seulement qu'un algorithme soit en mesure de vérifier que $y$ est une solution
de $x$.
Troisièmement,
un problème de la \textbf{classe $\sharp$P} est défini à partir d'un problème
$R = (X, Y)$ de la classe NP.
Le problème $N = (X, Y_R(X))$ de $\sharp$P correspondant est le problème de
recherche du nombre de solutions dans $R$ d'une entrée $x \in X$.

Évidemment, $\text{P} \subseteq \text{NP}$.
Par contre, 
il est toujours une question ouverte de déterminer si
$\text{P} \subset \text{NP}$ ou si $\text{P} = \text{NP}$.
Par contre,
la conjecture la plus acceptée est que $\text{P} \subset \text{NP}$.

Un problème $R = (X_R, Y)$ est \textbf{complet} pour une classe de complexité
si pour tout problème $S = (X_S, Y)$ de cette classe il existe une fonction 
$f: X_S \to X_R$ de complexité polynomiale.
En particulier,
un problème $R$ est \textbf{NP-complet} si un algorithme $A$ pour $R$
permet de construire, avec un surcout polynomial,
un algorithme pour résoudre un problème arbitraire de NP.

\chapter{Théorie des groupes}
\label{chap:theo_groupes}

Un \textbf{groupe} $(G, *)$ est un ensemble $G$ avec une opération $* : G \times G \to G$
satisfaisant les trois conditions suivantes.
L'opération $*$ est associative.
Il existe un élément identité $e$ tel que $e * g = g * e = g$ pour tous $g \in G$.
Chaque élement $g \in G$ a un inverse $g^{-1} \in G$ tel que $g * g^{-1} = g^{-1} * g = e$.
Par exemple $(\mathbb Z, +)$ est le groupe des nombres entiers avec l'opérateur d'addition.
De plus, un groupe est \textbf{abélien} si l'opération est commutative.

Les groupes considérés dans la thèse sont tous des groupes multiplicatifs.
Dans ce cas, l'opérateur est noté $\cdot$ et régulièrement omis.
Ainsi, $gh$ et $g \cdot h$ sont deux notations équivalentes pour la multiplication 
de deux éléments d'un groupe.
Comme l'opération est sous-entendu,
il est commun de noter un groupe $(G, \cdot)$ seulement à partir de l'ensemble $G$.

La \textbf{cardinalité} $|G|$ d'un groupe $G$ est le nombre d'éléments du celui-ci.
Les groupes considérés dans la thèse ont tous une cardinalité finie.
Un \textbf{sous-groupe} est un sous-ensemble $H \subseteq G$ satisfaisant les trois conditions
d'un groupe.
Je note alors $H \sqsubseteq G$.
Un sous-ensemble $S \subseteq G$ est un \textbf{générateur} d'un sous-groupe $H$ si $H$ est le plus petit
sous-groupe contenant $S$.
J'utilise $g(H)$ pour représenter un générateur de $H$ 
et $\langle S \rangle$ pour représenter le sous-groupe généré par $S$.
Les éléments d'un sous-ensemble $S \subseteq G$ sont \textbf{indépendants} si 
$|\langle T \rangle| > |\langle S \rangle|$ pour tous $T \subset S$.

Pour $H \sqsubseteq G$,
la \textbf{classe à gauche} d'un élément $g \in G$ est
\begin{align}
  gH = \qty{gh : h \in H}.
\end{align}
De même,
la \textbf{classe à droite} de $g$ est 
\begin{align}
  Hg = \qty{hg : h \in H}.
\end{align}
Si $Hg = gH$ pour tous $g \in G$,
alors $H$ est un \textbf{sous-groupe normal} et,
pour tous $f, g \in G$, l'opération
\begin{align}
  (fH)\cdot(gH) = (fg)H
\end{align}
entre les classes de $H$ respecte la définition d'un groupe.
Cela permet de former le \textbf{groupe quotient} 
\begin{align}
  G / H = \qty{gH : g \in G},
\end{align}
soit le groupe des classes de $H$.

La \textbf{conjugaison} de $g \in G$ et $H \sqsubseteq G$ est 
\begin{align}
  gHg^{-1} = \qty{ghg^{-1} : h \in H}.
\end{align}
Le \textbf{normalisateur} de $H$ est l'ensemble des éléments de $G$ qui laisse $H$
invariant après conjugaison.
C'est-à-dire, l'ensemble
\begin{align}
  \mathcal{N}_G(H) = \qty{g \in G : gHg^{-1} = H}.
\end{align}
Le \textbf{centralisateur} de $H$ est l'ensemble des éléments de $G$ qui commutent avec
tous les éléments de $H$, soit
\begin{align}
  \mathcal{C}_G(H) = \qty{g \in G : gh = hg,\, h \in H}.
\end{align}
Il est toujours vrai que $\mathcal{C}_G(H) \subseteq \mathcal{N}_G(H)$ puisque
si $g \in \mathcal{C}_G(H)$ alors $ghg^{-1} = hgg^{-1} = h$.

Soit un groupe $G$ tel que $fg = \pm gf$ pour tous $f,g \in G$
et $H \sqsubseteq G$ tel que $h \in H \implies -h \not\in H$,
alors $\mathcal{N}_G(H) = \mathcal{C}_G(H)$ puisque
\begin{align}
  ghg^{-1} = \pm hgg^{-1} = \pm h = h
\end{align}
pour tous $h \in H$.
Ceci est le cas pour le groupe stabilisateur présenté au chapitre~\ref{chap:construction_codes}
de cette thèse.
