\begin{comment}
\end{comment}

\appendix
% \renewcommand\chapterstring{Annexe}

\chapter{Complexité de calcul}
\label{chap:complexite_calcul}

\begin{itemize}
  \item Notation $O(\cdot)$ et $\Omega(\cdot)$
\end{itemize}

\chapter{Théorie des graphes}
\label{chap:theo_graphe}

\begin{itemize}
  \item Définition
  \item Graphes biparties et hypergraphes
\end{itemize}

\chapter{Théorie de l'information}
\label{chap:theo_info}

\begin{itemize}
  \item Entropie
  \item Information mutuelle
\end{itemize}

\chapter{Théorie des groupes}
\label{chap:theo_groupes}

Un \textbf{groupe} $(G, *)$ est un ensemble $G$ avec une opération $* : G \times G \to G$
satisfaisant les trois conditions suivantes.
L'opération $*$ est associative.
Il existe un élément identité $e$ tel que $e * g = g * e = g$ pour tous $g \in G$.
Chaque élement $g \in G$ a un inverse $g^{-1} \in G$ tel que $g * g^{-1} = g^{-1} * g = e$.
Par exemple $(\mathbb Z, +)$ est le groupe des nombres entiers avec l'opérateur d'addition.
De plus, un groupe est \textbf{abélien} si l'opération est commutative.

Les groupes considérés dans la thèse sont tous des groupes multiplicatifs.
Dans ce cas, l'opérateur est noté $\cdot$ et régulièrement omis.
Ainsi, $gh$ et $g \cdot h$ sont deux notations équivalentes pour la multiplication 
de deux éléments d'un groupe.
Comme l'opération est sous-entendu,
il est commun de noter un groupe $(G, \cdot)$ seulement à partir de l'ensemble $G$.

La \textbf{cardinalité} $|G|$ d'un groupe $G$ est le nombre d'éléments du celui-ci.
Les groupes considérés dans la thèse ont tous une cardinalité finie.
Un \textbf{sous-groupe} est un sous-ensemble $H \subseteq G$ satisfaisant les trois conditions
d'un groupe.
Je note alors $H \sqsubseteq G$.
Un sous-ensemble $S \subseteq G$ est un \textbf{générateur} d'un sous-groupe $H$ si $H$ est le plus petit
sous-groupe contenant $S$.
J'utilise $g(H)$ pour représenter un générateur de $H$ 
et $\langle S \rangle$ pour représenter le sous-groupe généré par $S$.
Les éléments d'un sous-ensemble $S \subseteq G$ sont \textbf{indépendants} si 
$|\langle T \rangle| > |\langle S \rangle|$ pour tous $T \subset S$.

Pour $H \sqsubseteq G$,
la \textbf{classe à gauche} d'un élément $g \in G$ est
\begin{align}
  gH = \qty{gh : h \in H}.
\end{align}
De même,
la \textbf{classe à droite} de $g$ est 
\begin{align}
  Hg = \qty{hg : h \in H}.
\end{align}
Si $Hg = gH$ pour tous $g \in G$,
alors $H$ est un \textbf{sous-groupe normal} et,
pour tous $f, g \in G$,
\begin{align}
  (fH)\cdot(gH) = (fg)H
\end{align}
défini une opération respectant la définition d'un groupe pour les classes de $H$.
Cela permet de former le \textbf{groupe quotient} 
\begin{align}
  G / H = \qty{gH : g \in G}.
\end{align}

La \textbf{conjugaison} de $g \in G$ et $H \sqsubseteq G$ est 
\begin{align}
  gHg^{-1} = \qty{ghg^{-1} : h \in H}.
\end{align}
Le \textbf{normalisateur} de $H$ est l'ensemble des éléments qui laisse $H$
invariant après conjugaison.
C'est-à-dire, l'ensemble
\begin{align}
  \mathcal{N}_G(H) = \qty{g \in G : gHg^{-1} = H}.
\end{align}
Le \textbf{centralisateur} de $H$ est l'ensemble des éléments qui commute avec
tous les éléments de $H$, soit
\begin{align}
  \mathcal{C}_G(H) = \qty{g \in G : gh = hg,\, h \in H}.
\end{align}
Il est toujours vrai que $\mathcal{C}_G(H) \subseteq \mathcal{N}_G(H)$ puisque
si $g \in \mathcal{C}_G(H)$ alors $ghg^{-1} = hgg^{-1} = h$.




\begin{itemize}
  \item Définition
  \item Centralisateur / Normalisateur
  \item Classe et groupe quotient
\end{itemize}


