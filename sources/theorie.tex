\begin{comment}
\end{comment}

\chapter{Théorie}

%-----------------------------------------------------------------------------%

% Ce qui suit n'est que du remplissage avec un petit exemple de ref/eqref
\section{Section}
\subsection{Exemple math \label{sec:newton}}

Newton a reçu -- ou pas -- une pomme sur la tête avant de déterminer, dans \cite{newton1687philosophiae}, que
\begin{align}
    \label{eq:newton}
    \vec F=m \vec a,
\end{align}
son équation la plus célèbre. Comme on le voit donc à l'équation~\eqref{eq:newton} de la section~\ref{sec:newton}, la force est proportionnelle à l'accélération.

On peut aussi exprimer la mécanique sous la forme lagrangienne. Pour des coordonnées généralisées $q$ et leur dérivées $\dot q$, le lagrangien est alors

\begin{align}
    \label{eq:lagrangien}
    L(q, \dot q, t) = T - V
\end{align}

et les équations du mouvement dérivent des équations de Lagrange 

\begin{align}
    \label{eq:equations_lagrange}
    \frac{\partial L}{\partial q} - \frac{\mathrm{d}}{\mathrm{d}t}\frac{\partial L}{\partial \dot q} = 0 \quad \forall q
    .   % Fin de la phase -> point
\end{align}

Ainsi, bien que ça ne soit pas évident de prime abord, les équations~\eqref{eq:newton} et~\eqref{eq:equations_lagrange} sont conceptuellement équivalentes!

\subsection{Exemple citation}
Citons aussi un vieil article \cite{andreev1964} et un autre \cite{robertson1929}. Maintenant citons les deux à la fois \cite{andreev1964, robertson1929}. Ensuite, citons plusieurs références avec un commentaire chacunes \cites[p.~1]{robertson1929}[chap.~2]{andreev1964}[\textsection 6]{newton1687philosophiae}. Finalement, citons les mêmes références, mais ne mettons des commentaires que lorsque nécessaire \cites{robertson1929}[chap.~2]{andreev1964}{newton1687philosophiae}.

\subsection{Parenthèses ajustables}
Voici des exemples d'utilisantion des paranthèses ajustables définies dans le package \emph{commandes}\footnote{Voici comment faire une note en bas de page. Voir le fichier \texttt{commandes.sty} pour des commandes personnalisées et personnalisables.}:
\begin{align}
    a&=\p{123}            \\
    a&=\p[]{123}          \\
    a&=\p[big]{123}       \\
    a&=\pc[Big]{123}      \\
    a&=\pc[bigg]{123}     \\
    a&=\pa[Bigg]{123}     \\
    a&=\pa[none]{123}     \\
    a&=\moy{\int t\dd t}  \\
    a&=\norm{\int t\dd t} \\
    a&=\abs[Bigg]{123}    \\
\end{align}

\subsection{Remplissage}
\kant[8-10]

