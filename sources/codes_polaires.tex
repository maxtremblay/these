\begin{comment}
\end{comment}

\chapter{Étude des codes polaires convolutifs}

Le problème de communication classique est définie comme suit.
Nous avons deux acteurs, 
que nous nommerons Arthur et Béatrice pour faire changement\footnote{
  Une recherche en ligne vous démontrera que beaucoup d'attention a été donnée 
  à Alice et Bob. 
},
qui veulent échanger de l'information.
Dans notre cas, 
supposons qu'Arthur cherchent à envoyer un message à Béatrice qui se trouve
des kilomètres plus loin.
Pour ce faire,
Arthur utilise son téléphone cellulaire et envoie un texto à Béatrice.
Une fois le message composé,
celui-ci est converti en une séquence de nombres binaires (des 0 et des 1)
qui sera transmise via des ondes radios jusqu'au téléphone de Béatrice
qui devra reconvertir la séquence de 0 et de 1 en message intelligible.

L'histoire ne s'arrête pas là puisque lors de la transmission du signal
dans l'atmosphère, il est fort probable que celui-ci soit corrompu en 
raison d'une interaction indésirée.
La solution à ce problème est de transmettre une séquence de nombres binaires
plus longue que ce qui est nécessaire en espérant que l'information supplémentaire
nous aide à retrouver le message initial en cas d'erreur.

Dans ce chapitre,
je vais d'abord reformuler ce problème dans un langage mathématique plus formel.  
Par la suite, 
je vais présenter les réseaux de tenseurs,
un outil mathématique qui va m'être très utile pour décrire les codes polaires 
ainsi qu'une généralisation de ceux, les codes polaires convolutifs.
Ces codes sont très importants puisque les codes polaires sont au coeur 
de la technologie de communication 5G et que toutes améliorations de ceux-ci
a un grand potentiel d'application.
À cet effet, le chapitre se termine par une présentation d'un article scientifique
dans lequel j'identifie les régimes où les performances des codes polaires convolutifs sont 
les plus impressionnantes.

\section{Communication classique et correction d'erreurs}



\section{Réseaux de tenseurs}

\section{Codes polaires}

\section{Codes polaires convolutifs}

\section{Article : Comparaison de la profondeur et la largeur des codes polaires convolutifs}
