\begin{comment}
\end{comment}

\Introduction   % Chapitre qui ne sera pas numéroté si IntroConcluSansNombre est Vrai

L'ordinateur et l'internet figurent parmi les technologies qui ont le plus impacté
notre mode de vie et l'organisation de nos sociétés.
En effet,
il est désormais possible d'accéder à une quantité phénoménale de connaissance,
de connecter avec une personne de l'autre côté du globe
et d'automatiser les taches du quotidient.
Tout cela avec une efficience qui peut sembler sans limite.

Cependant, 
bien que ces technologies offrent des performances impressionantes,
il existe des taches pour lesquelles le temps nécessaire peut s'approche ou
dépasse l'age actuel de l'univers.
Bien sur,
le temps d'exécution dépend de la taille de la tache a effectuer.
Par contre, 
la difficulté de différentes taches n'augmente pas de manière similaire.

Par exemple,
il sera deux fois plus long pour un ordinateur d'inverser l'ordre d'une liste 
si la longueur de cette dernière est doublée.
Par contre,
il ne suffit que d'ajouter un chiffre à un nombre pour que celui-ci soit deux 
fois plus long à factoriser~\cite{arora_computational_2009}.
Formellement,
on dit que la complexité du premier problème augmente linéairement avec la taille de l'entrée,
alors que celle du second problème augmente exponentiellement.

Les problèmes dont la complexité augmente exponentiellement requierent donc un temps de calcul
hors de porté, et ce, même pour des instances de tailles modestes.
Par contre,
cette façon de calculer la complexité suppose un modèle de calcul dit classique,
soit le modèle utilisé par les ordinateurs actuels.
Et,
comme vous vous en doutez probablement si vous lisez cette thèse,
le modèle de calcul classique n'est pas le seul modèle de calcul.

L'idée du calcul quantique est attribuée à Ed Fredkin et Richard Feynman et 
c'est ce dernier qui aurait présenté cette possibilité en 1981 lors d'une conférence au MIT~\cite{hoofnagle_birth_2021}.
Depuis,
le calcul quantique a fait beaucoup de chemin,
nottament après une publication de Peter Shor dans laquelle il introduit un 
algorithme pouvant factoriser un nombre avec une complixité polynomiale selon
le nombre de chiffres à l'aide d'un ordinateur quantique~\cite{shor_algorithms_1994}.
Dans ce cas,
doubler la longueur du nombre de fait que multiplier par huit le temps de calcul.
Cela est beaucoup plus efficace que la croissance exponentielle requise par un ordinateur classique.

Bien que ce résultat ne permet de résoudre tous les problèmes dont la complexité classique
est exponentielle,
il démontre tout de même que certains problèmes qui semblaient hors d'atteintes des ordinateurs
classiques seraient en théorie résolubles, en un temps raisonnable, par un ordinateur quantique.



