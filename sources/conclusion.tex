\begin{comment}
\end{comment}

\Conclusion % Chapitre qui ne sera pas numéroté si IntroConcluSansNombre est Vrai

Dans cette thèse,
j'ai présenté trois projets de recherche effectués lors de mon doctorat.
Tous ces projets traitent de la correction des erreurs,
mais dans des modèles de calcul différents.

Lors du premier projet,
j'ai mesuré l'impact de divers paramètres sur la performance et le temps de décodage
des codes polaires convolutifs.
Cela a permis d'identifier que les codes de profondeur et largeur deux offrent les
meilleures performances en fonction du temps de décodage.
Ce résultat pousse donc plus loin les performances des codes polaires standards
qui sont présentement utilisés dans le réseau de communication 5G.
Bien que les considérations d'ingénierie d'un réseau de communication sortent du cadre de cette thèse,
il est évident que toutes les améliorations à la correction d'erreurs peuvent réduire les 
couts d'implémentation d'un tel réseau.
En effet,
comme le taux d'erreur dépend de la distance entre les antennes,
un code qui permet de corriger plus d'erreurs permet également d'augmenter la distance
entre les antennes ce qui engendre une réduction du nombre nécessaire.

Le second projet que j'ai présenté est en réalité le dernier que j'ai effectué.
Dans le cadre de celui-ci,
je me suis intéressé à la construction de codes correcteurs quantiques.
La contribution la plus importante de ce projet réside dans la connexion
entre les problèmes de satisfaction de contraintes et la construction de codes.
Cela a permis de démontrer numériquement l'existence d'un seuil de satisfaisabilité
pour la probabilité d'inclusion d'une arête dans le graphe de support,
au-delà duquel il est facile de trouver des codes.
Il existe donc des régimes de paramètres pour lesquels il est aisé de construire des codes.
De plus,
j'ai poussé la méthode plus loin pour construire une famille de codes LDPC offrant des 
performances optimales pour le canal à effacement.
Il reste encore du chemin à parcourir,
mais cette approche offre la possibilité de découvrir des codes correcteurs pouvant être
adaptés aux contraintes physiques des ordinateurs quantiques à court et moyen terme.

Le troisième projet dont j'ai traité s'intéresse à la réalisation d'une architecture
permettant l'implémentation d'une mémoire quantique.
Des simulations numériques montrent également que cette approche permet de réduire significativement
le nombre de qubits nécessaires.
Bien sûr,
il reste de nombreux défis à résoudre pour implémenter cette architecture en laboratoire,
notamment l'implémentation de connexions à longue portée entre les qubits.
Néanmoins,
l'approche présentée permet d'éliminer les croisements entre les connexions en exploitant
un petit nombre de couches dans la troisième dimension.
Enfin,
les méthodes de traçage de graphes utilisées pour obtenir cette architecture 
ont le potentiel d'être appliquées pour contourner d'autres limitations à l'implémentation
du calcul tolérant aux fautes.

Dans l'ensemble de ces projets,
les modèles de bruits étudiés demeurent agnostiques de la technologie utilisée pour construire
les qubits.
Ce choix est fait puisqu'il est encore trop tôt pour choisir une technologie parmi toutes celles
proposées.
Il reste donc du travail à faire pour adapter les méthodes proposées dans cette thèse aux diverses 
technologies de qubits.
Cependant,
les travaux de cette thèse démontrent bien comment il est possible d'utiliser divers outils,
des réseaux de tenseurs aux méthodes de traçage de graphes en passant par les problèmes de satisfaction
de contraintes,
pour faciliter le design de systèmes quantiques tolérant aux fautes.
