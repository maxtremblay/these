\begin{comment}
\end{comment}

C'est ouvrage traite principalement de la protection de
l'information. 
Pas au sens que nous entendons souvent dans les médias de protection des renseignements privés,
mais plutôt au sens de robustesse face à la corruption des données.
En effet,
lorsque nous utilisons un cellulaire pour envoyé un texto,
plusieurs facteurs, 
comme les particules atmosphériques et l'interférence avec d'autres signaux,
peuvent modifier le message initial.
Si nous ne faisons rien pour protéger le signal,
il est peu problable que le contenu du texto reste inchangé
lors de la réception.

C'est ce problème qui a motivé le premier projet 
de recherche de cette thèse.
Sous la supervision du professeur David Poulin,
j'ai étudié une généralisation des codes polaires,
une technologie au coeur du protocol de communication de 5ième génération (5G).
Pour cela,
j'ai utilisé les réseaux de tenseurs, 
des outils mathématiques initialement développé pour étudier
les matériaux quantiques.
L'avantage de cette approche est qu'elle permet une représentation
graphique intuitive du problème, 
ce qui facilite grandement le développement des algorithmes.

Cette idée d'utiliser des outils mathématiques graphiques pour 
étudier des problèmes de protection de l'information
sera le fil conducteur pour le reste de la thèse.
Cependant, 
pour la suite, 
les erreurs n'affecteront plus des systèmes de communications classiques,
mais plutôt des systèmes de calcul quantique.
Et, comme nous le verrons dans cette thèse,
les systèmes quantiques sont naturellement beaucoup plus sensible aux erreurs.

À cet effet, 
j'ai effectué un stage au sein de l'équipe de Microsoft Research,
principalement sous la supervision de Micheal Beverland, 
lors duquel j'ai conçu des circuits permettant de mesurer un système quantique afin d'identifier
les potentiels fautes qui affectent celui-ci.
J'ai également proposée une architecture qui permetterait d'implémenter ces circuits
de façon réaliste en laboratoire.
Finalement, avec le reste de l'équipe, 
nous avons prouver mathématiquement que les circuits que j'ai développés sont optimaux.
L'ensemble de ces résultats sont inspirés de méthodes issues de la théorie des graphes.

J'ai terminé ma thèse sous la supervision du professeur Stefanos Kourtis.
Avec celui-ci,
j'ai créé une méthode, toujours basée sur une approche graphique, 
qui permet d'automatiquement concevoir de nouveaux protocoles de 
corrections des erreurs dans un système quantique.
Cela nous a permis de montrer qu'il est probablement beaucoup plus facile 
que ce que le croyait la communauté scientifique de concevoir de tels protocoles.

Vous aurez remarqué que lors de tous ces projets,
je n'ai jamais eu le même superviseur.
C'est pour cela que je qualifie ma thèse d'odyssée.
Celle-ci a été parsemée d'embuches.
D'abord avec le triste départ trop rapide du professeur David Poulin.
C'est suite à cela que j'ai postulé pour un stage au sein de l'équipe de Microsoft
dans le but d'avoir une nouvelle supervision.
Cependant, le stage c'est finalement déroulé en virtuel après avoir été repousssé
plusieurs fois en raison de la pandémie.
C'est après ce stage que je me suis finalement greffé à l'équipe du professeur Kourtis 
qui venait de démarrer son groupe.
Heureusement,
lors de toutes cette étape, 
je pouvais toujours compter sur le soutien de Guillaume Duclos-Ciani,
initialement professionel de recherche au sein du groupe de David Poulin.

En bref,
j'espère que vous aurez du plaisir à lire cette thèse même si celle-ci 
tire un peu dans tous les sens.
Elle réflète les connaissances et compétences que j'ai acquéries en travaillant sous 
la supervision de plusieurs mentors en plus de s'intéresser à l'un des problèmes
parmi les plus importants pour la réalisation des promesses de l'informatique quantique.

Bonne lecture!
