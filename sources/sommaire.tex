\begin{comment}
\end{comment}

\clearpage  % Mets le sommaire à la bonne page dans la TOC
\chapter*{Sommaire}
\addcontentsline{toc}{chapter}{Sommaire}

Cet ouvrage traite principalement de la protection de l'information. Non pas au
sens de protection des renseignements privés dont on entend souvent parler
dans les médias,
mais plutôt au sens de robustesse à la corruption des données.
En effet, lorsque nous utilisons un cellulaire pour envoyer un texto, plusieurs
facteurs, comme les particules atmosphériques et l'interférence avec d'autres
signaux, peuvent modifier le message initial. Si nous ne faisons rien pour protéger le signal, 
il est peu probable que le contenu du texto reste inchangé lors de la réception.

C'est ce problème qui a motivé le premier projet de recherche de cette thèse.
Sous la supervision du professeur David Poulin, 
j'ai étudié une généralisation des codes polaires,
une technologie au cœur du protocole de télécommunication de 5\textsuperscript{ième} génération (5G). 
Pour cela, j'ai utilisé les réseaux de tenseurs, outils
mathématiques initialement développés pour étudier les matériaux quantiques.
L'avantage de cette approche est qu'elle permet une représentation graphique
intuitive du problème, ce qui facilite grandement le développement des algorithmes.
À la suite de cela,
j'ai étudié l'impact de deux paramètres clés sur la performance des 
codes polaires convolutifs.
En considérant le temps d'exécution des protocoles,
j'ai identifié les valeurs de paramètres qui permettent de mieux protéger 
l'information à un cout raisonnable.
Ce résultat permet ainsi de mieux comprendre comment améliorer les performances
des codes polaires, ce qui a un grand potentiel d'application en raison
de l'importance de ces derniers.

Cette idée d'utiliser des outils mathématiques graphiques pour étudier des
problèmes de protection de l'information sera le fil conducteur pour le reste de
la thèse. Cependant, pour la suite, les erreurs n'affecteront plus des systèmes
de communications classiques, mais plutôt des systèmes de calcul quantique.
Et, comme je le présenterai dans cette thèse, les systèmes quantiques sont
naturellement beaucoup plus sensibles aux erreurs.

À cet égard, j'ai effectué un stage au sein de l'équipe de Microsoft Research,
principalement sous la supervision de Michael Beverland, au cours duquel j'ai conçu
des circuits permettant de mesurer un système quantique afin d'identifier les
potentielles fautes qui affectent celui-ci. 
Avec le reste de l'équipe, nous avons prouvé mathématiquement
que les circuits que j'ai développés sont optimaux.
Ensuite, j'ai proposé une architecture pour implémenter ces circuits
de façon plus réaliste en laboratoire
et les simulations numériques que j'ai effectuées ont démontré des résultats 
prometteurs pour cette approche. 
D'ailleurs, ce résultat a été accueilli avec grand intérêt par la communauté
scientifique et a été publié dans la prestigieuse revue \textit{Physical Review Letters}.
Pour complémenter ce travail,
j'ai collaboré avec l'équipe de Microsoft pour démontrer analytiquement 
que les architectures actuelles d'ordinateurs quantiques reposant sur des connexions
locales entre les qubits ne suffiront pas pour la réalisation d'ordinateurs de grandes
tailles protégés des erreurs.
L'ensemble de ces résultats sont inspirés de méthodes issues de la théorie des graphes
et plus particulièrement des méthodes de représentation des graphes dans un espace en 
deux dimensions.
L'utilisation de telles méthodes pour le design de circuits et d'architectures quantiques
est également une approche novatrice.

J'ai terminé ma thèse sous la supervision du professeur Stefanos Kourtis.
Avec celui-ci, j'ai créé une méthode, 
fondée sur la théorie des graphes et des méthodes d'informatique théorique,
qui permet de concevoir automatiquement de nouveaux protocoles de 
correction des erreurs dans un système quantique.
La méthode que j'ai conçue repose sur la résolution d'un problème 
de satisfaction de contraintes.
Ce type de problème est généralement très difficile à résoudre.
Cependant,
il existe pour ces derniers un paramètre critique. 
En variant ce paramètre, 
le système passe d'une phase où les instances sont facilement résolubles 
vers une phase où il est facile de montrer qu'il n'y pas de solution.
Les problèmes difficiles sont alors concentrés autour de cette transition.
À l'aide d'expériences numériques,
j'ai montré que la méthode proposée a un comportement similaire.
Cela permet de montrer qu'il existe un régime où il est beaucoup plus facile 
que ce que le croyait la communauté de concevoir des protocoles de corrections
des erreurs quantiques.
De plus,
autant que je sache,
l'article qui a résulté de ce travail est le premier qui met de l'avant
ce lien entre la construction de protocoles de corrections des erreurs,
les problèmes de satisfaction de contraintes et les transitions de phases.

Vous aurez remarqué que lors de tous ces projets,
je n'ai jamais eu le même superviseur.
C'est pour cela que je qualifie ma thèse d'odyssée.
Celle-ci a été parsemée d'embuches.
D'abord avec le triste départ trop rapide du professeur David Poulin.
C'est à la suite de son décès que j'ai postulé un stage au sein de l'équipe de Microsoft
dans le but d'obtenir une nouvelle supervision.
Cependant, le stage s'est finalement déroulé en virtuel après avoir été repoussé
plusieurs fois en raison de la pandémie.
C'est après ce stage que je me suis finalement greffé à l'équipe du professeur Kourtis 
qui venait de démarrer son groupe.
Heureusement,
lors de toutes ces étapes, 
je pouvais toujours compter sur le soutien de Guillaume Duclos-Cianci,
initialement professionnel de recherche au sein du groupe de David Poulin.

En bref,
j'espère que vous aurez du plaisir à lire cette thèse même si celle-ci 
peut paraitre légèrement éclectique.
Elle reflète les connaissances et compétences que j'ai acquises en travaillant sous 
la supervision de plusieurs mentors en plus de s'intéresser à l'un des problèmes
parmi les plus importants pour réaliser les promesses de l'informatique quantique.

Bonne lecture !
