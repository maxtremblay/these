\begin{comment}
\end{comment}

%=============================================================================%

\makeatletter   % Permet d'accèder aux variables @

%=============================================================================%

    % PDFA-1b/Hyperref
    \iftoggle{generatePDFa}
    {
        % PDF/A stuff (experimental)
        \usepackage[a-1b]{pdfx}
    }
    {
        \newcommand{\sep}{, }
    }%
        \begin{filecontents*}[overwrite]{\jobname.xmpdata}
\Title          {\@title}
\Author         {\@author}
\Subject        {\ResumeCourt}
\Org            {\Organisation}
\Keywords       {\MotsClefs}
\PublicationType{book}
        \end{filecontents*}%
        % Couleurs moins intenses, https://ethanschoonover.com/solarized/
        \definecolor{solblue}{HTML}{268bd2}
        \definecolor{solred}{HTML}{dc322f}
        \definecolor{solviolet}{HTML}{6c71c4}
        \definecolor{solmagenta}{HTML}{d33682}
        \definecolor{solcyan}{HTML}{2aa198}
        \definecolor{solorange}{HTML}{cb4b16}
        \definecolor{solyellow}{HTML}{b58900}
        \definecolor{solgreen}{HTML}{859900}
        % Hyperrefs
        \usepackage{hyperref}
        \hypersetup{
            pdfauthor={\@author},
            pdftitle={\@title},
            pdfsubject={\ResumeCourt},
            pdfkeywords={\MotsClefs},
            % pdfa,
            colorlinks=true,
            breaklinks=true,
            urlcolor=solmagenta,
            linkcolor=solblue,
            citecolor=solcyan,
            bookmarksopen=true,
            unicode=true
        }
        \usepackage[hyperpageref]{backref}      % Backrefs!
        \renewcommand{\backref}[1]{[cf.~p.~#1]} % Volé cette ligne à Samuel Boutin
        \usepackage{bookmark}
        \usepackage{cmap} % Doit être après hyperref pour être compatible avec pdf-a

%=============================================================================%

    \iftoggle{useCustomFonts}
    {   
        % Deux prochaines lignes -> Pagella et Mathpazo
        %\usepackage{mathpazo} % utilise Palatino pour les mathématiques (mettre en premier)
        %\usepackage{tgpagella} % utilise la police TeX Gyre Pagella

        % Deux prochaines lignes -> New Century et Fourier
        %\usepackage{newcent}
        %\usepackage{fouriernc}
        
        \usepackage{newcent}
        \usepackage{fouriernc}
        \renewcommand{\dagger}{\text{\textdagger}} % Millennial missing dagger fix
        \renewcommand{\iint}{\int\!\!\int}  % Better spacing
        \renewcommand{\iiint}{\int\!\!\int\!\!\int}
        \renewcommand{\iiiint}{\int\!\!\int\!\!\int\!\!\int}
    }
    {}

%=============================================================================%
    
    \iftoggle{AuteureFemme}
    { \newcommand{\monsieurMadame}{Mme.} }    
    { \newcommand{\monsieurMadame}{M.} }

%=============================================================================%

    \iftoggle{MemoirePasThese}
    {   % Si c'est un mémoire
        \newcommand{\documentPresente}{Mémoire présenté}
        \newcommand{\leDocument}{le mémoire}
        \newcommand{\leGrade}{maître ès science (M.Sc.)}
    }
    {   % Si c'est une thèse
        \newcommand{\documentPresente}{Thèse présentée}
        \newcommand{\leDocument}{la thèse}
        \newcommand{\leGrade}{docteur ès science (Ph.D.)}
    }

%=============================================================================%
    
    % Gestion de la version électronmique vs celle imprimée.
    \iftoggle{VersionLivre}
    {   % On fait la version imprimée!

        % On utilise un fontsize plus petit(10pt vs 12pt).
        \let\small\relax
        \let\footnotesize\relax
        \let\scriptsize\relax
        \let\tiny\relax
        \let\large\relax
        \let\Large\relax
        \let\LARGE\relax
        \let\huge\relax
        \let\Huge\relax
        \input{size10.clo}  % Ajuste le fontsize ET les marges
        \geometry{twoside=true, bottom=2.54cm} % L'ordre est important, pour les marges/fontsize.
        %\widowpenalty5000  % Décommenter s'il y a trop de ligne veuves/orphelines.
        %\clubpenalty5000   


        % On enlève la numérotation des pages vides
        \let\origdoublepage\cleardoublepage
        \newcommand{\clearemptydoublepage}{%
            \clearpage%
            {\pagestyle{empty}\origdoublepage}%
            }
        \let\cleardoublepage\clearemptydoublepage
        
        % Permet de mettre un page blanche seulement dans la version imprimée
        \newcommand{\autoPageBlancheLivre}{\clearpage\null\thispagestyle{empty}}

        \iftoggle{LivrePGChVide}
        {
            \let\stdchapter\chapter
            \renewcommand\chapter{\clearpage\null\thispagestyle{empty}\stdchapter}
            \renewcommand{\autoPageBlancheLivre}{}

            \let\stdpart\part
            \renewcommand\part{\clearpage\null\thispagestyle{empty}\stdpart}

            \renewcommand\@endpart{\vfil
                          \if@twoside
                            \null
                            \thispagestyle{empty}%
                            \newpage
                          \fi
                          \if@tempswa
                            \twocolumn
                          \fi}
        }{}

        % Les hyperliens n'ont pas desoins d'être colorés
        \hypersetup{hidelinks}
        % On affiche les DOI dans la biblio -> Pratique en version imprimée
        \bibliographystyle{config/nature-fr-showdoi}
        % On garde les entêtes dans la bibliographie
        \newcommand{\bibpagestyle}{}
    }
    {   % S'il y a un côté, on fait la version électronique.
        \geometry{letterpaper, lmargin=1.25in, rmargin=1.25in,
                  tmargin=1.5in, bmargin=1.0in, twoside=false}
        \iftoggle{ForceEntete}
        {
            % On garde les entêtes dans la bibliographie
            \newcommand{\bibpagestyle}{}
        }
        {
        % Prochaines lignes enlèvent l'entête
            \renewcommand{\chaptermark}[1]
                {\markboth{{\thechapter. #1}}{}}
            \renewcommand{\sectionmark}[1]{}
            % On enlève les entêtes dans la bibliographie
            \newcommand{\bibpagestyle}{\pagestyle{plain}}
        }

        % On n'affiche pas les DOI dans la biblio -> Plus propre
        \bibliographystyle{config/nature-fr}
        %\bibliographystyle{unsrt-fr} % unsrt partiellement traduit. Préférer nature.
        
        % Permet de mettre un page blanche seulement dans la version imprimée
        \newcommand{\autoPageBlancheLivre}{}
    }
    \iftoggle{IntroConcluSansNombre}
    {
        \newcommand{\Introduction}{\chapter*{Introduction}
            \addcontentsline{toc}{chapter}{Introduction}}
        \newcommand{\Conclusion}{\chapter*{Conclusion}
            \addcontentsline{toc}{chapter}{Conclusion}}
    }
    {
        \newcommand{\Introduction}{\chapter{Introduction}}
        \newcommand{\Conclusion}{\chapter{Conclusion}}
    }


%=============================================================================%

\makeatother    % Plus d'accès aux variables @
